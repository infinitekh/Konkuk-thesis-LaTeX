\documentclass[final,doctor,korean,pdfdoc]{konkukthesis}
% \input{package}
% \documentclass{konkukreport}
 \titles[korean]{3-spin 모형변형 }
\titles[english]{Three spin model}
\titles[subtitle]{}
% 
\author[korean]{김 혁}
\author[hanja]{金 爀}
\author[english]{KIM, Hyeok}
\advisor{여 교 수}{Prof. YEO}{}
\department[physics]
% 
% 
% 
% 
\referee[1]{여두령}
\referee[2]{송차령}
\referee[3]{박삼령}
\referee[4]{이사령}
\referee[5]{김오령}
% % 
\gradyear{2020}{2}
\submityear{2019}{11}
\appryear{2018}{12}

 \keywords[korean]{안녕, 키워드}
 \keywords[english]{hello, namer}
% \nonstopmode
\usepackage{amsmath}
\usepackage{amsfonts}
\usepackage{amssymb}


\begin{document}
\makecontents
% 	\tableofcontents
% 	\listoftables
% 	\listoffigures
% 	\appendix
  \begin{abstractEng}
In a sunny outdoor ceremony at the Imperial Palace, Trump shook hands with Emperor Naruhito and his wife, engaging warmly as he became the first foreign leader to meet the new monarch. Later he reviewed a cordon of Japanese troops and greeted yellow-hatted schoolchildren waving flags.
Those were more buttoned-up occasions than the round of golf, sumo wrestling tournament and hibachi dinner the President enjoyed Sunday. In manners-conscious Japan, Trump's interactions with the monarch and his wife were closely scrutinized. And his talks with Abe will come after a Sunday tweet from Trump threw into question the leaders' alignment on North Korea.
  \end{abstractEng}
\main
\chapter{별헤는밤 }
% \normalsize
여야 4당, 최대한 기간 단축 시도할듯..한국당 반발로 난항 예상
특위·법사위·본회의 단계별로 고비 전망..본회의서 `변심표' 나올 수도
국회 `선거제, 공수처법, 검경수사권 조정' 갈등 (PG) [장현경, 이태호 제작] 일러스트
% 
% 국회 `선거제, 공수처법, 검경수사권 조정' 갈등 (PG) [장현경, 이태호 제작] 일러스트
% (서울=연합뉴스) 차지연 기자 = 여야 4당(더불어민주당·바른미래당·민주평화당·정의당)의 강력한 드라이브와 자유한국당의 격렬한 반대가 뒤얽히며 우여곡절을 겪은 끝에 선거제·개혁법안 패스트트랙(신속처리안건) 지정이 완료됐다.
% 
% 이제 선거제 개혁안과 고위공직자범죄수사처(공수처)법 2건, 검경수사권 조정안은 패스트트랙 열차를 타고 최장 330일의 여정을 시작하게 된다.
% 
% 여야 4당은 이 기간을 최소 180일까지 줄이겠다는 계획이다. 그러나 공수처법 2건을 조율해 단일안을 만들어야 하고, 한국당의 강한 반발도 예상돼 본회의 처리까지 원활한 진행을 장담할 수는 없다.
% 
% 최장 330일 패스트트랙…여야 4당, 180일까지 단축 시도할 듯
% 
% 국회법 85조 2항에 따르면 패스트트랙으로 지정된 법안은 상임위원회 심사 180일, 법제사법위원회 체계·자구 심사 90일, 본회의 부의 후 상정까지 60일의 기간을 거치게 된다.
% 
% 여야 4당은 이를 최대한 단축하는 것을 목표로 하고 있다.
% 
% 현 상황에서 선거제와 공수처법, 검경수사권 조정안 패스트트랙 기간은 180일까지 줄일 수 있을 것으로 보인다.
% 
% 선거제 개혁안을 논의할 정치개혁특별위원회 위원장은 정의당 심상정 의원, 공수처법, 검경수사권 조정안을 논의할 사법개혁특별위원회 위원장은 민주당 이상민 의원으로 모두 패스트트랙에 찬성하는 입장이다.
% 
% 정개특위와 사개특위 모두 위원장이 회의를 열어 여야 4당 의원들만으로 법안 의결을 서두르면 180일 상임위(특위) 심사 기간은 획기적으로 줄어들 수 있다.
% 
% 그러나 한국당이 상임위(특위) 심사 기간을 확보하기 위해 국회법 57조 2항에 규정된 안건조정위원회 구성을 요청할 가능성이 있다. 안건조정위원회는 구성일로부터 90일간 활동하도록 돼 있다.
% 
% 이 때문에 여야 4당이 속도를 높이더라도 상임위(특위) 심사 기간은 최소 90일이 될 것으로 보인다.
% 
% 이후 밟게 되는 법사위 체계·자구 심사 기간 90일은 법사위원장이 한국당 여상규 의원인 점을 고려하면 사실상 단축이 불가능할 것으로 전망된다.
% 
% 본회의 부의 후 상정까지의 기간은 문희상 국회의장이 법안이 부의되자마자 상정한다면 60일 전부를 단축할 수 있다.
% 
% 결국 여야 4당이 한국당의 협조 없이 최대한 속도를 낸다면 특위 심사 90일, 법사위 심사 90일 등 180일만 거치는 것으로 기간을 줄일 수 있다는 계산이 나온다.
% 
% 여기에 특위 활동시한 연장 여부가 변수로 떠오를 수 있다.
% 
% 사개특위와 정개특위는 활동시한이 6월 말까지인데, 이를 연장하지 않으면 선거제 개혁안은 행정안전위원회로, 공수처법·검경수사권 조정안은 법사위로 각각 넘어가게 된다.
% 
% 특위 심사 기간과 특위 종료 후 행안위·법사위 심사 기간을 합쳐 상임위 심사 기간으로 정해져 있는 180일을 맞추게 되고 안건조정위원회 구성 시 90일도 그대로 적용될 것으로 전망된다.
% 
% 다만 이 경우 행안위는 민주당 인재근 의원이라 위원장이라 심사 기간 단축에 큰 무리가 없어 보이지만, 여상규 의원이 위원장인 법사위에서는 단축이 쉽지 않아질 것으로 예상된다.
\section{section}
선거제 개혁안과 공수처법·검경수사권 조정안 모두 패스트트랙 여정이 절대 순탄치는 않을 것으로 보인다.
\subsection{subsection}
특위 심사와 법사위 심사, 본회의 처리까지 단계마다 만만찮은 고비가 있다.

한국당이 특위 심사 중 소극적 참여로 논의를 묶어두는 것부터 의결 저지까지 갖가지 방법을 동원하면 여야 4당의 심사 기간 단축 계획은 차질을 빚을 수밖에 없다.

여야 4당은 이번 패스트트랙 지정을 `논의의 시작'으로 삼겠다는 생각이다.

일단 난항 끝에 함께 올린 두 개의 공수처 법안의 접점을 찾는 것이 제1과제다.

민주당 백혜련 의원 발의 법안과 바른미래당 권은희 의원 발의 법안은 수사 대상과 제한적 기소권 등 큰 틀에서 내용이 비슷하지만 기소심의위원회 설치 여부, 공수처장 임명 방식 등이 달라 조율이 필요하다.

단일한 안을 마련해 본회의 표결에 부치기 위해 민주당과 바른미래당은 심사 과정에서 상당한 진통을 겪을 것으로 보인다.

여기에 패스트트랙을 진행하면서도 한국당과 합의해 법안을 처리하려는 시도 역시 이어가겠다는 것이 여야 4당의 입장이다.

정개특위 소속인 민주당 김종민 의원은 ``가능한 한 빨리 심사해야겠지만 한국당에서 논의를 하겠다는 의사를 밝히면 그렇게 해야 한다''며 ``심사 기간을 최대한 단축하려고 하겠지만 상황은 지켜봐야 한다''고 말했다.

사개특위 소속인 민주당 박범계 의원은 ``패스트트랙에 오른 지금부터 논의가 시작되는 것''이라며 ``한국당이 끝까지 반대하느냐, 수정제의를 하느냐, 협상 여지를 보이느냐 등 여러 변수가 있다''고 말했다.

특위 심사와 법사위 심사 과정을 무사히 지나 본회의에 상정된다고 하더라도 통과를 장담할 수는 없다.

선거제 개혁안의 경우 의원 개개인의 `밥그릇'이 걸린 만큼, 패스트트랙을 함께 추진한 여야 4당 의원 중에서도 `변심'을 하는 경우가 생길 수 있기 때문이다.

아무리 패스트트랙 열차가 속도를 내도 최소 6개월이 걸리는 만큼, 정계 상황이 그사이 크게 달라질 수도 있다.

민주당 이종걸 의원은 페이스북 글에서 ``패스트트랙은 법안 내용을 결정하는 것이 아니라 법안처리 절차를 결정하는 것에 불과하다''며 ``패스트트랙에 찬성한 의원이 그 (법안) 내용에 반드시 동의해야 하는 것은 아니다''라고 지적했다.

이 의원은 ``최소 6개월 이상, 최장 330일이 걸리는데 그때도 합의 처리한 4당이 그대로 존속해 있을까? 본회의 부의 무렵에는 정치 지형이 달라질 가능성이 높다''며 ```소속 정당이 달라질 수도 있는' 의원들이 과거 소속 당에서 한 결정을 번복하는 데 무슨 큰 부담을 가지겠느냐''고 분석했다.

총선이 임박해 선거법과 공수처법 등이 본회의에 상정되면 한국당이 필리버스터(합법적 의사진행 방해)로 처리를 지연시킬 수도 있다는 전망도 나온다.
\begin{table}
 \caption{이건위에다가 캡션}
 안녕하세요 
\end{table}

\clearpage

\begin{abstractKor}
\indent
가자 똥 치우러2시간전 \par
한일전 승리의 날. 기분 좋다. \par
답글43댓글 찬성하기1623댓글 비추천하기154 
\textbf{}
거센-바람2시간전 \par
연합 너 삐딱하게 기사 쓸래 콱.. \par
답글16댓글 찬성하기650댓글 비추천하기23 

movieboy2시간전 \par 
기레기야\, 험로를 원하는가 \par 
답글2
\end{abstractKor}


\bibliographystyle{asa}
\bibliography{testbibliography}

\clearpage
\AppendixTitleToToc
\AttachAppendixTitleToSecnum
\appendix
\renewcommand\thechapter{\Alph{APPchapter}}
\renewcommand\thesection{\thechapter.\arabic{APPsection}}
\renewcommand\thesubsection{\thesection.\arabic{APPsubsection}}
\setcounter{APPchapter}{0}
\chapterstyle{appendixdefault}

% \renewcommand*\prechapternum{\chapnamefont 부록~}
\renewcommand*\prechapternum{\chapnamefont \appendixname\space}
% \renewcommand*\postchapternum{ }
\renewcommand*\postchapternum{}

% \renewcommand*\printchapternum{\chapnumfont\thechapter}


\renewcommand\hchaptertitlehead{\appendixname\,\thechapter\,}
 
\chapter{자료}
\section{섹션}
\subsection{서브섹션}
\chapter{예제}
 
\begin{align}
 a
 \\b
 \\c
\end{align}

\chapter{자료}

% \thebibliography
%
\end{document}
